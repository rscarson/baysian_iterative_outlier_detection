\section{Conclusion}

We have presented the Bayesian Iterative Outlier Detection (BIOD) algorithm, a method for adaptively determining the number of iterations required to achieve a desired confidence level in randomized testing. By combining Bayesian statistics with iterative refinement based on observed failure rates, BIOD improves upon traditional fixed-iteration approaches.

Empirical evaluation shows that BIOD is robust to poorly specified parameters and accurately models failure rates under independent, normally distributed errors. It balances computational cost with confidence in results, making it practical for resource-constrained settings such as CI pipelines. Future work could explore extending BIOD to correlated or non-normal error models, further broadening its applicability.

\bibliographystyle{plain}
\bibliography{references}